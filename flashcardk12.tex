\cardfrontfoot{Kapitel 12}

\begin{flashcard}[Teori]{Begrund at magnesium(II) har en mindre ionradius end natrium(I)}
Begge ioner har den samme elektronkonfiguration $1s^{2}2s^{2}2p^{6}$\\ \vspace{7pt}
Dog har magnesium �n proton mere end natrium. Det betyder, at magnesium kan ud�ve en st�rre tiltr�kkende kraft p� elektronerne s�ledes at de befinder sig t�ttere p� kernen.
\end{flashcard}

\begin{flashcard}[Reaktion]{Opskriv reaktionen mellem en (for det meste) intert gas og magnesium metal}
\ce{3Mg + N2 -> Mg3N2}
\end{flashcard}

\begin{flashcard}[Egenskab]{Angiv hvilke af jordalkalimetallerne der er opl�selige med \ce{CO3^{2-}}, \ce{PO4^{3-}}, \ce{SO4^{2-}} og \ce{OH-}}
\begin{tabular}{ r|c|c|c|c| }
 \multicolumn{1}{r}{}
  & \multicolumn{1}{c}{\ce{CO3^{2-}}}
  & \multicolumn{1}{c}{\ce{PO4^{3-}}}
  & \multicolumn{1}{c}{\ce{SO4^{2-}}}
  & \multicolumn{1}{c}{\ce{OH-}}
  
   \\
 \cline{2-5}
 \ce{Mg^{2+}}& & & Opl�selig & \\
 \cline{2-5}
 \ce{Ca^{2+}} & & & (Opl�selig) & (Opl�selig) \\
 \cline{2-5}
 \ce{Sr^{2+}} & & & & Opl�selig \\
 \cline{2-5}
 \ce{Ba^{2+}} & & & & Opl�selig \\
 \cline{2-5}
 \end{tabular}
\end{flashcard}

\begin{flashcard}[Reaktion]{Vis med reaktionsskema at berylliumoxid er amfotert}
\ce{BeO + 2H3O+ + H2O -> [Be(OH2)4]^{2+}}\\ \vspace{7pt}
\ce{BeO + 2OH- + H2O -> [Be(OH)4]^{2-}}
\end{flashcard}

\begin{flashcard}[Teori]{Begrund hvorfor beryllium har tendens til at danne covalente forbindelser}
Beryllium er relativt elektronegativ. Man kan forudsige bindingskarakter ud fra elektronegativitet. Et eksempel er \ce{BeCl2}. Forskellen mellem elektronegativitet for disse er $3.16-1.57=1.59$ hvilket svarer til en pol�r kovalent binding.
\end{flashcard}

\begin{flashcard}[Struktur]{Optegn strukturen af \ce{[Be(OH2)4]^{2+}}}
\schemestart
$\chemleft[\chemfig{Be(-[2]\ce{OH2})(-[5]\ce{H2O})(<[:-50]\ce{OH2})(<:[:-12]\ce{OH2})}\chemright]^{2+}$
\schemestop
\end{flashcard}

\begin{flashcard}[Fremstilling]{Hvordan findes magnesium i naturen?}
Magnesium findes i naturen som \ce{KMgCl3\cdot 6H2O}, \ce{CaMg(CO3)2} og \ce{MgSO4\cdot 7H2O}
\end{flashcard}

\begin{flashcard}[Reaktion]{Opskriv reaktion for forbr�nding af magnesium metal med oxygen henholdsvis carbondioxid}
\ce{2Mg + O2 -> 2MgO} \\ \vspace{7pt}
\ce{2Mg + CO2 -> 2MgO + C}
\end{flashcard}

\begin{flashcard}[Fremstilling]{Beskriv den industrielle fremstilling af magnesium}
\ce{Ca(OH)2 + Mg^{2+} -> Mg(OH2)(s) + Ca^{2+}} \\
\ce{Mg(OH)2 + 2HCl -> MgCl2(aq) + 2H2O} \\
Elektrolyse af \ce{MgCl2} giver \ce{Mg} ved katoden og chlorgas ved anoden. Chlorgas kan genbruges til at danne saltsyre.
\end{flashcard}

\begin{flashcard}[Reaktion]{Hvad sker der n�r \ce{CaCO3} opvarmes?}
\ce{CaCO3 -> CaO + CO2}
\end{flashcard}

\begin{flashcard}[Reaktion]{Opskriv hovedkomponenterne i klinker samt reaktionen hvorved cement h�rder}
Hovedkomponenterne i klinker er \ce{Ca3SiO5}, \ce{Ca3Al2O6} og \ce{Ca4Al2Fe2O10}.\\ \vspace{7pt}
\ce{2Ca3SiO5 + 7H2O -> Ca3Si2O7\cdot 4H2O + 3Ca(OH)2}\\
\ce{Ca(OH)2 + CO2 -> CaCO3 + H2O}
\end{flashcard}

\begin{flashcard}[Reaktion]{Opskriv den kemiske formel for gips og for det tilsvarende hemihydrat}
Gips: \ce{CaSO4\cdot 2H2O}\\ \vspace{7pt}
Tilsvarende hemihydrat: \ce{CaSO4\cdot $\frac{1}{2}$H2O}
\end{flashcard}

\begin{flashcard}[Fremstilling]{Opskriv reaktionen for dannelse af calciumcarbid}
\ce{CaO + 3C ->[\text{$\Delta$}] CaC2 + CO}
\end{flashcard}

\begin{flashcard}[Reaktion]{Opskriv calciumcarbids reaktion med vand henholdsvis nitrogen}
\ce{CaC2 + 2H2O -> Ca(OH)2 + C2H2}\\ \vspace{7pt}
\ce{CaC2 + N2 -> CaCN2 + C}
\end{flashcard}