\cardfrontfoot{Kapitel 21}

\begin{flashcard}[Teori]{Forventes 4-6 periode overgangsmetallerne at være lav spin eller høj spin?}
Lav spin da CFSE vokser ned gennem perioderne.
\end{flashcard}

\begin{flashcard}[Teori]{Hvad forstås ved \textit{lanthanoid contraction}?}
Elektronerne i $f$ orbitaler skærmer i meget ringe grad for de ydre elektroner som så oplever en stærkere tiltrækning fra kernen hvilket fører til en lavere ionradius. Derfor har overgangsmetallerne i 6. periode næsten samme radius og dermed ladningstæthed som dem i 5. periode.
\end{flashcard}

\begin{flashcard}[Fremstilling]{Hvordan fremstilles sølv industrielt?}
\ce{2AgS + 8CN- + O2 + 2H2O -> 4[Ag(CN)2]- + 2S + 4OH-}\\\vspace*{0.5cm}
\ce{2[Ag(CN)2]- + Zn -> 2Ag + [Zn(CN)4]^{2-}}
\end{flashcard}

\begin{flashcard}[Reaktion]{Der tilføjes sølvioner til en opløsning der enten indeholder iodid, bromid eller chlorid ioner. Hvordan kan man de eneklte ioner?}
Sølvchlorid er opløseligt i fortyndet ammoniak mens sølvbromid er opløseligt i koncentreret ammoniak. Sølviodid er ikke opløseligt i ammoniak.\\\vspace*{0.5cm}
\ce{AgCl + 2NH3 -> [Ag(NH)2]+ + Cl-}\\
\ce{AgBr + 2NH3[konc] -> [Ag(NH)2]+ + Br-}
\end{flashcard}