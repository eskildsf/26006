\cardfrontfoot{Kapitel 11}
\begin{flashcard}[Reaktion]{Beskriv henholdsvis lithiums reaktion med atmosfæren (oxygen og kuldioxid) samt alkalimetallernes reaktion med vand}

\ce{4Li + O2 -> 2Li2O}\\
\ce{Li2O + CO2 -> Li2CO3}\\ \vspace{7pt}
\ce{2K + 2H2O -> 2KOH + H2}\\
ligeledes for de andre.
\end{flashcard}

\begin{flashcard}[Egenskab]{Forklar hvorfor \ce{Li+} er exceptionel god til at koordinere vand}
\ce{Li+} har godt nok kun én positiv ladning. Til gengæld er Van der Walls radius af ionen relativt lille hvilket fører til en relativt høj ladningstæthed (ladning pr. volumen). Det er ladningstætheden der afgører ionens evne til at koordinere vand.
\end{flashcard}

\begin{flashcard}[Egenskab]{Opskriv alkalimetallernes flammefarver}
\begin{tabular}{ l l }
Lithium & \color{red}Rød \\
Natrium & \color{yellow}Gul \\
Kalium & \color{violet}Lilla \\
Rubidium & \color{red}Rød-violet \\
Cesium & \color{blue}Blå
\end{tabular}
\end{flashcard}

\begin{flashcard}[Egenskab]{Hvilken sammenhæng er der mellem opløseligheden af et salt, kationens radius og anionens radius?}
Kationer og anioner af nogenlunde samme størrelse vil have lettere ved at skabe et stabilt gitter (krystal, bundfald) end kationer og anioner med vidt forskellige størrelser. Derfor vil salte af ioner med stor størrelsesmæssig forskel ofte være let opløselige. Eksempelvis \ce{LiI}.
\end{flashcard}

\begin{flashcard}[Reaktion]{Opskriv reaktionen mellem nitrogen og et alkalimetal der har en rød flammefarve og høj ladningstæthed. Opskriv da produktets reaktion med vand.}
\ce{6Li + N2 -> 2Li3N} \\ \vspace{7pt}
\ce{Li3N + 3H2O -> NH3 + 3LiOH}
\end{flashcard}

\begin{flashcard}[Anvendelse]{Beskriv med ord og reaktionsskema hvorledes lithium indgår i genopladelige Lithium-Ion batterier}
Anoden består af \ce{LiCoO2(s)} og katoden af grafit. Ved opladning bevæger \ce{Li+} ioner sig fra anoden til katoden hvor de interkaleres i grafit katoden.\\ \vspace{7pt}
\ce{LiCoO2 -> Li_{($1-x$)}CoO2 + $x$Li+ + $x$e-} \\
\ce{C + $x$Li+ + $x$e- -> Li_{$x$}C} \\ \vspace{7pt}
Den modsatte reaktion finder sted ved afladning.
\end{flashcard}

\begin{flashcard}[Anvendelse]{Beskriv med reaktionsskema hvorledes lithium indgår i ikke-genopladelige batterier}
De har alle lithiums ionisering (anodereaktionen) til fælles: \ce{Li -> Li+ + e-}\\ \vspace{7pt}
Katodereaktionerne varierer batterityperne imellem. Her er tre forskellige batteritypers katodereaktion: \\
\ce{2SOCl2 + 4e- -> 4Cl- + SO2 + S} \\
\ce{SO2Cl2 + 2e- -> 2Cl- + SO2} \\
\ce{2SO2 + 2e- -> S2O4^{-2}}
\end{flashcard}

\begin{flashcard}[Fremstilling]{Opskriv hvordan titanium fremstilles industrielt}
\ce{TiCl4 + 4Na -> 4NaCl + Ti}
\end{flashcard}

\begin{flashcard}[Fremstilling]{Opskriv hvordan natrium og kalium fremstilles industrielt}
Natrium fremstilles ved elektrolyse af natriumchloridopløsning \\ \vspace{7pt}
\ce{Na+ + e- -> Na} \\
\ce{2Cl- -> Cl2 + 2e-} \\ \vspace{7pt}
Kalium fremstilles ved følgende reaktion ved $850\,^{\circ}{\rm C}$ \\
\ce{Na(l) + KCl(l) -> K(g) + NaCl(l)}
\end{flashcard}

\begin{flashcard}[Fremstilling]{Opskriv hvordan natriumhydroxid fremstilles industrielt}
Elektrolyse af natriumchloridopløsning \\ \vspace{7pt}
\ce{2H2O + 2e- -> H2 + 2OH-} \\
\ce{2Cl- -> Cl2 + 2e-} \\ \vspace{7pt}
De dannede hydroxid ioner er forhindret i at kommer i kontakt med chlorgassen af et diaphragm hvor natriumchloridopløsningen kan passere.
\end{flashcard}

\begin{flashcard}[Reaktion]{Opskriv oxiderne af alkalimetallerne samt deres reaktion med vand}
\ce{Li2OH}, \ce{Na2O2}, \ce{KO2}. \\ \vspace{7pt}
\ce{Li2O + H2O -> 2LiOH} \\
\ce{Na2O2 + 2H2O -> 2NaOH + H2O2} \\
\ce{2KO2 + 2H2O -> 2KOH + H2O2 + O2}
\end{flashcard}

\begin{flashcard}[Anvendelse]{Beskriv med reaktionsligninger hvorledes \ce{KO2} kan bruges til at oplagre kuldioxid}
\ce{4KO2 + 2CO2 -> 2K2CO3 + 3O2} \\
\ce{K2CO3 + H2O + CO2 -> 2KHCO3}
\end{flashcard}

\begin{flashcard}[Generelt]{Er dioxid(2-)ionen para- eller diamagnetisk? Begrund med MO teori.}
$2p$ elektronerne danner følgende molekylorbitaler
\begin{MOdiagram}[style=fancy,labels,AO-width=8pt,labels-fs=\footnotesize]
\atom[\ce{O-}]{left}{2p={;pair,pair,up}}
\atom[\ce{O-}]{right}{2p={;pair,pair,up}}
\molecule[\ce{O2^{-2}}]{2pMO={1.3,.4;pair,pair,pair,pair,pair,}}
\end{MOdiagram}\\[-5pt]Diamagnetisk, ingen uparrede elektroner.
\end{flashcard}

\begin{flashcard}[Generelt]{Er dioxid(1-)ionen para- eller diamagnetisk? Begrund med MO teori.}
$2p$ elektronerne danner følgende molekylorbitaler
\begin{MOdiagram}[style=fancy,labels,AO-width=8pt,labels-fs=\footnotesize]
\atom[\ce{O-}]{left}{2p={;pair,pair,up}}
\atom[\ce{O-}]{right}{2p={;pair,up,up}}
\molecule[\ce{O2^{-2}}]{2pMO={1.3,.4;pair,pair,pair,pair,up,}}
\end{MOdiagram}\\[-5pt]Paramagnetisk, uparrede elektroner i $2\pi^{*}$.
\end{flashcard}

\begin{flashcard}[Reaktion]{Opskriv reaktionen mellem aluminium metal og hydroxidionen}
\ce{2Al + 2OH- + 6H2O -> 2[Al(OH)4]- + 3H2}
\end{flashcard}

\begin{flashcard}[Reaktion]{Hvad sker der med en natriumhydroxidopløsning uden låg?}
\ce{OH- + CO2 -> HCO3- }
\end{flashcard}

\begin{flashcard}[Reaktion]{Salte af alkalimetalionerne samt ammoniumionen er normalt letopløselige. Som de eneste er alkalimetalionerne f.eks. letopløselige som carbonater. Opskriv reaktioner hvorved \ce{Na+}, \ce{K+} og \ce{NH4+} kan bundfældes}
Natrium\\
\ce{Na+ + [Sb(OH)6]- -> Na[Sb(OH)6](s)} \\ \vspace{7pt}
Kalium og ammonium\\
\ce{3K+ + [Co(NO)6]^{3-} -> K3[Co(NO)6](s)}\\
\ce{3NH4+ + [Co(NO)6]^{3-} -> (NH4)3[Co(NO)6](s)}
\end{flashcard}

\begin{flashcard}[Anvendelse]{Beskriv med reaktionsskemaer hvorledes natriumbicarbonat anvendes i bagepulver}
Bagepulver består af \ce{NaHCO3} samt \ce{Ca(H2PO4)2}.\\ \vspace{7pt}
\ce{2NaHCO3 + Ca(H2PO4)2 ->[\text{$\Delta$}] NaHPO4 + CaHPO4 + 2CO2 + 2H2O}
\end{flashcard}

\begin{flashcard}[Reaktion]{Hvad sker der med natriumbicarbonat når det opvarmes?}
\ce{2NaHCO3 -> Na2CO3 + CO2 + H2O}
\end{flashcard}

\begin{flashcard}[Egenskab]{Beskriv med ord og reaktionsskema hvad der sker når et alkalimetal, i dette tilfælde natrium,  opløses i ammoniak}
\ce{Na(s) -> Na+(ammoniak) + e^{-}(ammoniak)}\\
Opløsningen vil have en dyb blå farve når den er tynd og en bronze farve når det er koncentreret. Med tiden vil natrium reagere med ammoniak og danne natriumamid\\
\ce{2Na+ + 2NH3 + 2e- -> 2NaNH2 + H2}
\end{flashcard}

\begin{flashcard}[Fremstilling]{Hvordan findes kaliumchlorid i naturen og hvordan udvindes det?}
\ce{KCl} findes bl.a. som \ce{KMgCl3\cdot 6H2O} samt \ce{MgSO4\cdot H2O}. Udvindes v. 3 forskellige metoder.\\
\begin{itemize}
\item Udnyt forskellige opløsligheder af saltene ved at opløse dem. Energikrævende at fordampe vand.
\item Opløs i saltlage. Blæs luft igennem. \ce{KCl} sidder fast på boblernes overflade som opfanges.
\item Elektrostatisk proces. Mal krystaller til pulver og giv dem ladning via. friktion. De kan herefter adskilles.
\end{itemize}
\end{flashcard}

\begin{flashcard}[Fremstilling]{Fra hvilket mineral og hvordan udvindes \ce{Na2CO3}?}
Trona: \ce{Na2CO3\cdot NaHCO3 \cdot 2H2O}\\ \vspace{7pt}
Opvarmning, rekrystallisation, opvarmning\\
\ce{2Na2CO3\cdot NaHCO3 \cdot 2H2O ->[\text{$\Delta$}] 3Na2CO3 + 5H2O + CO2}\\
Natriumcarbonat genopløses hvorved faste urenheder filtreres fra. \ce{Na2CO3\cdot H2O} opnås ved tørring.\\
\ce{Na2CO3\cdot H2O ->[\text{$\Delta$}] Na2CO3 + H2O(g)}
\end{flashcard}

\begin{flashcard}[Fremstilling]{Beskriv hvorledes \ce{Na2CO3} kan fremstilles ud fra Solvay processen}
\ce{2NaCl + CaCO3 <=> Na2CO3 + CaCl2}
\end{flashcard}

\begin{flashcard}[Anvendelse]{Beskriv med reaktionsskema hvorledes \ce{Na2CO3} anvendes i produktionen af glas.}
\ce{Na2CO3 + $x$SiO2 ->[\text{$\Delta$}] Na2O\cdot $x$SiO2 + CO2 }
\end{flashcard}