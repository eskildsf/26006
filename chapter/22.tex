\cardfrontfoot{Kapitel 22}

\begin{flashcard}[Struktur]{Giv reaktionsligningen for forbrænding af zink i chlorgas}
\ce{Zn + Cl2(g) -> ZnCl2(g)}
\end{flashcard}

\begin{flashcard}[Fremstilling]{Giv reaktionsligningerne for industriel fremstilling af zink}
\ce{2ZnS + 3O2 ->[\text{$\Delta$}] 2ZnO + 2SO2}\\\vspace*{0.5cm}
\ce{ZnO + C ->[\text{$\Delta$}] Zn + CO}
\end{flashcard}

\begin{flashcard}[Egenskab]{Forklar hvorfor zink kan beskytte fjern mod korrosion}
Reduktionspotentialet for zink er lavere end det er for jern. Derfor korroderer zink først hvilket efterlader jern intakt.
\end{flashcard}

\begin{flashcard}[Egenskab]{Hvordan kan \ce{Zn(OH)2} bringes i opløsning?}
Ved tilsætning af base i form af hydroxidioner eller ammoniak.\\\vspace*{0.5cm}
\ce{Zn(OH)2 + 2OH- -> [Zn(OH)4]^{2-}}\\
\ce{Zn(OH)2 + 4NH3 -> [Zn(NH3)4]^{2+} + 2OH-}
\end{flashcard}

\begin{flashcard}[Fremstilling]{Opskriv to metoder til fremstilling af zinkoxid}
\ce{2Zn + O2 -> 2ZnO}\\\vspace*{0.5cm}
\ce{ZnCO3 ->[\text{$\Delta$}] ZnO + CO2}
\end{flashcard}

\begin{flashcard}[Anvendelse]{Opskriv halvcellereaktionerne i et NiCad batteri}
\ce{Cd + 2OH- -> Cd(OH)2 + 2e-}\\\vspace*{0.5cm}
\ce{2NiO(OH) + 2H2O + 2e- -> 2Ni(OH)2 + 2OH-}
\end{flashcard}

\begin{flashcard}[Fremstilling]{Angiv med reaktionsligning hvordan kviksølv fremstilles industrielt}
\ce{HgS + O2 ->[\text{$\Delta$}] Hg + SO2}
\end{flashcard}

\begin{flashcard}[Fremstilling]{Hvordan kan man fremstille kviksølv(II)chlorid og kviksølv(I)chlorid?}
\ce{Hg + Cl2(g) -> HgCl2}\\\vspace*{0.5cm}
\ce{2HgCl2 + SnCl2 -> SnCl4 + Hg2Cl2}\\
Tilsættes overskud af tin(II)chlorid fås kviksølv\\
\ce{Hg2Cl2 + SnCl2 -> SmCl4 + 2Hg}
\end{flashcard}

\begin{flashcard}[Reaktion]{Hvilken reaktion finder sted når kviksølvoxid opvarmes kraftigt?}
\ce{2HgO ->[\text{$\Delta$}] 2Hg + O2}
\end{flashcard}

\begin{flashcard}[Anvendelse]{Giv halvcellereaktionerne der finder sted i et kviksølv batteri}
\ce{Zn + 2OH- -> Zn(OH)2 + 2e-}\\\vspace*{0.5cm}
\ce{HgO + H2O + 2e- -> Hg + 2OH-}
\end{flashcard}

\begin{flashcard}[Egenskab]{Kobber(I), guld(I) og \ce{Hg2^{2+}} ionen har tendens til at disproportionere. Giv reaktionsligningerne}
\ce{2Cu+ -> Cu2+ + Cu}\\\vspace*{0.5cm}
\ce{3Au+ -> 2Au + Au^{3+}}\\\vspace*{0.5cm}
\ce{Hg2^{2+} <=> Hg + Hg^{2+}}\\
Da ovenstående er en ligevægt kan den forskydes mod højre ved at fælde kviksølv(II) ionerne med sulfid.
\end{flashcard}

\begin{flashcard}[Egenskab]{Opskriv de tungtopløselige hydroxider af $d$ metallerne samt hvorvidt de er amfotere eller ej}
Ikke amfotere: \ce{Mn(OH)2}, \ce{MnO(OH)}, \ce{Fe(OH)2}, \ce{FeO(OH)}, \ce{Ni(OH)2}, \ce{NiO(OH)}, \ce{Cd(OH)2}\\\vspace*{0.5cm}
Amfotere: \ce{Co(OH)2}, \ce{Cu(OH)2}, \ce{Zn(OH)2}\\
Der dannes tetraedriske komplekser når ovenstående reagerer med stærk base.
\end{flashcard}