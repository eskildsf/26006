\cardfrontfoot{Kapitel 9}
\begin{flashcard}[Trend]{Hvilken type bindinger danner grundstofferne i 5. hovedgruppe?}
Nitrogen og fosfor laver covalente bindinger. Arsen laver netværk-covalente bindinger. Antimon og bismuth laver metalliske bindinger.
\end{flashcard}

\begin{flashcard}[Trend]{Hvilken type bindinger danner grundstofferne i 2. periode?}
\begin{tabular}{ c | c | c | c | c | c | c | c }
\ce{Li} & \ce{Be} & \ce{B} & \ce{C} & \ce{N2} & \ce{O2} & \ce{F2} & \ce{Ne} \\ \hline
M & M & NC & NC & C & C & C & C
\end{tabular}\\ \vspace{7pt}
M = metallisk, NC = netværk covalent, C = covalent
\end{flashcard}

\begin{flashcard}[Trend]{Hvilken type bindinger danner grundstofferne i 3. periode?}
\begin{tabular}{ c | c | c | c | c | c | c | c }
\ce{Na} & \ce{Mg} & \ce{Al} & \ce{Si} & \ce{P4} & \ce{S8} & \ce{Cl2} & \ce{Ar} \\ \hline
M & M & M & NC & C & C & C & C
\end{tabular}\\ \vspace{7pt}
M = metallisk, NC = netværk covalent, C = covalent
\end{flashcard}

\begin{flashcard}[Trend]{Hvilken bindingstype er der tale om i de højeste flourider af grundstofferne i 2. periode?}
\begin{tabular}{ c | c | c | c | c | c }
\ce{LiF} & \ce{BeF2} & \ce{BF3} & \ce{CF4} & \ce{NF3} & \ce{OF2} \\ \hline
I & NC & C & C & C & C
\end{tabular}\\ \vspace{7pt}
I = ionisk, NC = netværk covalent, C = covalent
\end{flashcard}

\begin{flashcard}[Trend]{Hvilken bindingstype er der tale om i de højeste flourider af grundstofferne i 3. periode?}
\begin{tabular}{ c | c | c | c | c | c | c }
\ce{NaF} & \ce{MgF2} & \ce{AlF3} & \ce{SiF4} & \ce{PF5} & \ce{SF6} & \ce{ClF5} \\ \hline
I & I & NC & C & C & C & C
\end{tabular}\\ \vspace{7pt}
I = ionisk, NC = netværk covalent, C = covalent
\end{flashcard}

\begin{flashcard}[Trend]{Hvilken bindingstype er der tale om i de højeste oxider af grundstofferne i 2. periode?}
\begin{tabular}{ c | c | c | c | c | c }
\ce{Li2O} & \ce{BeO} & \ce{B2O3} & \ce{CO2} & \ce{N2O5} & \ce{F2O} \\ \hline
I & I & NC & C & C & C
\end{tabular}\\ \vspace{7pt}
I = ionisk, NC = netværk covalent, C = covalent
\end{flashcard}

\begin{flashcard}[Trend]{Hvilken bindingstype er der tale om i de højeste oxider af grundstofferne i 3. periode?}
\begin{tabular}{ c | c | c | c | c | c | c }
\ce{Na2O} & \ce{MgO} & \ce{Al2O3} & \ce{SiO2} & \ce{P4O10} & \ce{(SO3)3} & \ce{Cl2O7} \\ \hline
I & I & I & NC & C & C & C
\end{tabular}\\ \vspace{7pt}
I = ionisk, NC = netværk covalent, C = covalent
\end{flashcard}

\begin{flashcard}[Trend]{Hvilken bindingstype er der tale om i hydriderne af grundstofferne i 2. periode?}
\begin{tabular}{ c | c | c | c | c | c | c }
\ce{LiH} & \ce{(BeH2)_{$x$}} & \ce{B2H6} & \ce{CH4} & \ce{NH3} & \ce{H2O} & HF \\ \hline
I & NC & C & C & C & C & C
\end{tabular}\\ \vspace{7pt}
I = ionisk, NC = netværk covalent, C = covalent
\end{flashcard}

\begin{flashcard}[Trend]{Hvilken bindingstype er der tale om i hydriderne af grundstofferne i 3. periode?}
\begin{tabular}{ c | c | c | c | c | c | c }
\ce{NaH} & \ce{MgH2} & \ce{(AlH3)_{$x$}} & \ce{SiH4} & \ce{PH3} & \ce{H2S} & \ce{HCl} \\ \hline
I & I & NC & C & C & C & C
\end{tabular}\\ \vspace{7pt}
I = ionisk, NC = netværk covalent, C = covalent
\end{flashcard}

\begin{flashcard}[Trend]{Angiv syre/base egenskaberne af de højeste oxider af grundstofferne i 3. periode}
\begin{tabular}{ c | c | c | c | c | c | c }
\ce{Na2O} & \ce{MgO} & \ce{Al2O3} & \ce{SiO2} & \ce{P4O10} & \ce{(SO3)3} & \ce{Cl2O7} \\ \hline
B & B & A & S & S & S & S
\end{tabular}\\ \vspace{7pt}
B = basisk, S = sur, A = amfoter
\end{flashcard}

\begin{flashcard}[Trend]{Angiv syre/base egenskaberne af de højeste oxider af grundstofferne i 5. hovedgruppe}
\begin{tabular}{ c | c | c | c | c }
\ce{N2O5} & \ce{P4O10} & \ce{As2O3} & \ce{Sb2O3} & \ce{Bi2O3} \\ \hline
S & S & A & A & B
\end{tabular}\\ \vspace{7pt}
B = basisk, S = sur, A = amfoter
\end{flashcard}