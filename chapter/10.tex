\cardfrontfoot{Kapitel 10}
\begin{flashcard}[Fremstilling]{Hvorledes kan \ce{H2} fremstilles industrielt og i laboratoriet?}
Industrielt:\\ \ce{CH4 + H2O -> CO + 3 H2} \\ \ce{CO + H2O ->[\text{$\Delta$}] CO2 + H2} \\ \ce{K2CO3 + CO2 + H2O -> 2KHCO3} \\ \vspace{7pt}
I laboratoriet: \\ \ce{Zn(s) + 2HCl -> ZnCl2 + H2} \\ \vspace{7pt} samt ved elektrolyse i begge tilfælde:\\
\ce{2 H2O + 2e- -> 2 OH- + H2}\\
\ce{6H2O -> 4H3O+ + O2 + 4e-}
\end{flashcard}

\begin{flashcard}[Trend]{Stiger eller falder reaktiviteten mellem \ce{H2} og halogenerne ned gennem 7. hovedgruppe?}
Reaktiviteten mellem dihydrogen og halogenerne falder ned gennem 7. hovedgruppe.
\end{flashcard}

\begin{flashcard}[Reaktion]{Beskriv hvordan \ce{H2} kan anvendes som reduktionsmiddel}
\ce{H2} kan anvendes på organiske forbindelser:\\
\begin{align*}
\ce{H2\OX{rf1,\ox{-2,\ce{C}}}=\OX{rf2,\ox{-2,\ce{C}}}H2 + \OX{of1,\ox{0,\ce{H2}}} -> \OX{oe1,\ox{+1,\ce{H3}}}\OX{re1,\ox{-3,\ce{C}}}-\OX{re2,\ox{-3,\ce{C}}}\OX{oe2,\ce{H3}}}
\redox(of1,oe2){\small oxidation}
\redox(of1,oe1){}
\redox(rf1,re2)[][-1]{\small reduktion}
\redox(rf2,re1)[][-1]{}
\end{align*} \\ \vspace{7pt}
samt uorganiske, herunder metaloxider:\\
\begin{align*}
\ce{\OX{rf1,\ox{+2,\ce{Cu}}} O + \OX{of1,\ox{0,\ce{H2}}} -> \OX{re1,\ox{0,\ce{Cu}}} + \OX{oe1,\ox{+1,\ce{H2}}} O}
\redox(of1,oe1){\small oxidation}
\redox(rf1,re1)[][-1]{\small reduktion}
\end{align*}
\end{flashcard}

\begin{flashcard}[Trend]{Hvorledes kan hydriderne af grundstofferne i det periodiske system karakteriseres som henholdsvis ioniske, kovalente eller metalliske?}
Hydriderne af grundstofferne i 1. og 2. hovedgruppe kan karakteriseres som ioniske. \\
Hydriderne af overgangsmetallerne kan karakteriseres som metalliske. \\
Hydriderne af grundstofferne i 3. til 7. hovedgruppe kan karakteriseres som covalente.
\end{flashcard}

\begin{flashcard}[Egenskab]{Begrund hvorfor vands og flussyres kogepunkt er væsentligt højere end forventet}
Intermolekylære hydrogenbindinger.
\end{flashcard}

\begin{flashcard}[Reaktion]{Færdiggør og afstem\\ \vspace{7pt}
\ce{H2 + Na -> NaH}\\
\ce{H2 + F2 -> HF}\\
\ce{H2 + O2 -> H2O}\\
\ce{H2 + N2 -> NH3}\\
\ce{H2 + CuO ->[\text{$\Delta$}] Cu}}

\ce{H2 + 2Na -> 2NaH}\\
\ce{H2 + F2 -> 2HF}\\
\ce{2H2 + O2 -> 2H2O}\\
\ce{3H2 + N2 -> 2NH3}\\
\ce{H2 + CuO ->[\text{$\Delta$}] Cu + H2O}
\end{flashcard}